\section{Goals}\label{sec:goals}
The goal of this thesis is to build on the work done by Singh et al. described in Section~\ref{sec:sofar}.
To further complete the integration of SQL/PGQ into DuckDB several UDFs need to be implemented. 
The next step after being able to compute the reachability will be to compute the minimal length of a path between any two nodes. 
The paths can either have a weight equal to 1, or a custom weight which is a positive number. 
In either case, the minimal path length can be computed using either Bellman-Ford or Dijkstra's algorithm. 
Once computing this is possible, we will work on retrieving all the nodes and edges contained in the shortest / cheapest path and return this in a query.

Another goal of this thesis is to identify and implement various optimizations in DuckDB especially related to graph queries. 
One of these optimizations has already been identified and considers the case when multiple identical hash tables are built due to duplicate join statements.
These duplicate joins can occur in plain SQL queries, though are more common in SQL/PGQ queries due to the common occurrence of many-to-many relationships in graphs.
For example, first joining a person table with a knows table (for person1\_id), after which the person table is again joined with the knows table (for person2\_id). 
In both instances, the same hash table is built for an equivalent join, resulting in equivalent hash tables, which causes redundant work. 
This is wasted computation time and storage, given that one hash table would suffice for both joins.  
The optimization will be to identify whenever these identical joins are done and ensure that only the minimally required amount of hash tables are built. 

The research questions of this thesis will be: 
\begin{enumerate}
    \item How can shortest and cheapest paths best be implemented using MS-BFS?
    \begin{enumerate}
        \item What are the trade-offs between using Bellman-Ford and Dijkstra's algorithms?
        \item How can the number of states during computation be minimized to allow for optimal execution? 
    \end{enumerate}
    \item What are the performance bottlenecks in SQL/PGQ?
    \begin{enumerate}
        \item How can these performance bottlenecks be experimentally evaluated? 
    \end{enumerate}
\end{enumerate}

% As previously mentioned, work is already being conducted on integrating SQL/PGQ into DuckDB by Singh et al.~\cite{sqlpgq-duckdb}, though the work will not lead to complete integration. 
% Therefore, the main goal of this thesis is to further integrate SQL/PGQ into DuckDB. Before that can be done, the current state of integration needs to be clear, as well as what more needs to be done in terms of integration.
% For example, it is known that reachability can be computed. However, it is unknown whether this computation can be further optimized. The next step will be to compute the path length and eventually return the path as a result of the query. 

% The ultimate goal is to allow the integration to coexist as a part of the DuckDB code base, however, it is also possible that the integration will exist as an extension module to DuckDB.

% Another goal of this thesis is to optimize the query transformation in DuckDB. More specifically, whenever a query contains a join on the same table ($Table A \Join Table A$), identical hash tables are built, one for each entry of the table in the query. In most cases, this is wasted computation as the resulting hash tables will be equivalent. An optimization could to detect whenever this happens, and create a single hash table that is usable by both tables.

% \textcolor{red}{
% All in all, this leads to the following main research question: "How can SQL/PGQ best be further integrated into DuckDB?" 
% This research question will be answered in combination with related sub-questions that are related. 
% These include: 
% \begin{itemize}
%     \item What is the current state of integration of SQL/PGQ in DuckDB? 
%     \item What work needs to be performed to further integrate SQL/PGQ in DuckDB? 
%     \item How can the current integration of SQL/PGQ in DuckDB be further optimized? 
%     \item How should the integration of SQL/PGQ be analysed through empirical evaluations?
% \end{itemize}}