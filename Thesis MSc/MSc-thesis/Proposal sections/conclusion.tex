\section{Conclusion}\label{sec:conclusion}
In this thesis we will be working on implementing the minimal path length as a UDF in DuckDB. This will be done using the batched Bellman-Ford algorithm, and/or Batched Dijkstra's algorithm developed by Then et al. The goal of this is to further complete the integrating of SQL/PGQ in DuckDB. Alongside, we will identify and implement optimizations that are specifically useful for graph-like queries. One of such optimizations is the shared hash join, which eliminates the need of building duplicate hash tables. 

% This thesis proposal introduces the topic of integrating SQL/PGQ in the DBMS DuckDB. Given the rising popularity of graph based systems and the approval of the SQL/PGQ standard, implementing the standard in DuckDB is an interesting challenge. Work has been done on integrating SQL/PGQ~\cite{sqlpgq-duckdb}, this thesis aims to complete that work. Additionally, the thesis will focus on optimizations which can be done on queries containing joins on equivalent tables, something especially prevalent in tables containing vertices.  
